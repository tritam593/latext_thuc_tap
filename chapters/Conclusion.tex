Như vậy, báo cáo đã giới thiệu một số khái niệm cơ bản về lý thuyết nhóm, lý thuyết biểu diễn và equivariance, tạo nền tảng toán học cho việc xây dựng các mạng nơ-ron equivariant. Báo cáo đã đi sâu vào phân tích Tensor Field Network (TFN), một kiến trúc mạng nơ-ron đặc biệt hiệu quả trong việc xử lý dữ liệu điểm 3D. Cấu trúc của TFN, bao gồm các phép tích chập điểm, self-interaction và lớp phi tuyến, được thiết kế để đảm bảo tính equivariant đối với các phép biến đổi hình học như phép quay và tịnh tiến.

Báo cáo cũng đã nêu ra một số dạng dữ liệu đặc trưng trong GDL, từ dữ liệu đồ thị quen thuộc trong mạng xã hội và hóa học đến dữ liệu đa tạp phức tạp như hình dạng 3D và protein. Chúng ta đã tìm hiểu cách các mạng nơ-ron đồ thị (GNN) giải quyết các thách thức trong việc biểu diễn và xử lý dữ liệu đồ thị, đồng thời xem xét các trường hợp đặc biệt của GNN như DeepSets và Transformer.

Đặc biệt, báo cáo đã phân tích các thách thức và phương pháp xử lý dữ liệu đa tạp. Việc định nghĩa phép tích chập trên đa tạp, tính equivariant của nó đối với các phép biến đổi hình học, và các ứng dụng trong computer graphic và protein modeling được trình bày một cách chi tiết.

Báo cáo đã đề xuất một hướng nghiên cứu mới về ứng dụng Geometric Deep Learning trong lĩnh vực xử lý ngôn ngữ tự nhiên (NLP), cụ thể là cho AMR (Abstract Meaning Representation). Ý tưởng về việc xây dựng một không gian xác suất của các đồ thị AMR, cùng với việc phân tích các phép biến đổi giữa chúng, mở ra những hướng nghiên cứu đầy tiềm năng cho việc biểu diễn và xử lý ngữ nghĩa.

Thực nghiệm phân loại hình 3D sử dụng TFN trên tập dữ liệu ModelNet10 đã cho thấy hiệu quả của mạng nơ-ron equivariant trong việc xử lý dữ liệu 3D. Mặc dù kết quả đạt được chưa tối ưu do hạn chế về dữ liệu và phần cứng, nhưng nó đã chứng minh được tiềm năng của GDL trong việc giải quyết các bài toán thực tế.

Cuối cùng, Geometric Deep Learning là một lĩnh vực nghiên cứu sôi động với tiềm năng ứng dụng rộng lớn trong nhiều lĩnh vực khác nhau. Báo cáo này đã cung cấp một cái nhìn tổng quan về các khái niệm, phương pháp và ứng dụng của GDL, đồng thời đề xuất một hướng nghiên cứu mới đầy hứa hẹn trong lĩnh vực NLP.